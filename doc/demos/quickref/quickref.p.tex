%%
%% Automatically generated file from Doconce source
%% (http://code.google.com/p/doconce/)
%%
% #ifdef PTEX2TEX_EXPLANATION
%%
%% The file follows the ptex2tex extended LaTeX format, see
%% ptex2tex: http://code.google.com/p/ptex2tex/
%%
%% Run
%%      ptex2tex myfile
%% or
%%      doconce ptex2tex myfile
%%
%% to turn myfile.p.tex into an ordinary LaTeX file myfile.tex.
%% (The ptex2tex program: http://code.google.com/p/ptex2tex)
%% Many preprocess options can be added to ptex2tex or doconce ptex2tex
%%
%%      ptex2tex -DMINTED -DPALATINO -DA6PAPER -DLATEX_HEADING=traditional myfile
%%      doconce ptex2tex myfile -DMINTED -DLATEX_HEADING=titlepage
%%
%% ptex2tex will typeset code environments according to a global or local
%% .ptex2tex.cfg configure file. doconce ptex2tex will typeset code
%% according to options on the command line (just type doconce ptex2tex to
%% see examples).
% #endif

% #ifndef LATEX_HEADING
% #define LATEX_HEADING "doconce_heading"
% #endif

% #ifndef PREAMBLE
% #if LATEX_HEADING == "Springer_collection"
% #undef PREAMBLE
% #else
% #define PREAMBLE
% #endif
% #endif


% #ifdef PREAMBLE
%-------------------- begin preamble ----------------------

\documentclass[%
oneside,                 % oneside: electronic viewing, twoside: printing
final,                   % or draft (marks overfull hboxes)
10pt]{article}

\listfiles               % print all files needed to compile this document

% #ifdef A4PAPER
\usepackage[a4paper]{geometry}
% #endif
% #ifdef A6PAPER
% a6paper is suitable for mobile devices
\usepackage[%
  a6paper,
  text={90mm,130mm},
  inner={5mm},           % inner margin (two sided documents)
  top=5mm,
  headsep=4mm
  ]{geometry}
% #endif

\usepackage{relsize,epsfig,makeidx,color,setspace,amsmath,amsfonts}
\usepackage[table]{xcolor}
\usepackage{bm,microtype}
\usepackage{ptex2tex}

% #ifndef MOVIE
% #define MOVIE "media9"
% #endif

% #if MOVIE == "media9"
\usepackage{media9}
% #elif MOVIE == "movie15"
\usepackage{movie15}
% #elif MOVIE == "multimedia"
\usepackage{multimedia}
% #elif MOVIE == "href-run"
% #endif

% #ifdef MINTED
\usepackage{minted}
\usemintedstyle{default}
% #endif

% #ifdef XELATEX
% xelatex settings
\usepackage{fontspec}
\usepackage{xunicode}
\defaultfontfeatures{Mapping=tex-text} % To support LaTeX quoting style
\defaultfontfeatures{Ligatures=TeX}
\setromanfont{Kinnari}
% Examples of font types (Ubuntu): Gentium Book Basic (Palatino-like),
% Liberation Sans (Helvetica-like), Norasi, Purisa (handwriting), UnDoum
% #else
%\usepackage[latin1]{inputenc}
\usepackage[utf8]{inputenc}
% #ifdef HELVETICA
% Set helvetica as the default font family:
\RequirePackage{helvet}
\renewcommand\familydefault{phv}
% #endif
% #ifdef PALATINO
% Set palatino as the default font family:
\usepackage[sc]{mathpazo}    % Palatino fonts
\linespread{1.05}            % Palatino needs extra line spread to look nice
% #endif
% #endif

% Hyperlinks in PDF:
\definecolor{linkcolor}{rgb}{0,0,0.4}
\usepackage[%
    colorlinks=true,
    linkcolor=linkcolor,
    urlcolor=linkcolor,
    citecolor=black,
    filecolor=black,
    %filecolor=blue,
    pdfmenubar=true,
    pdftoolbar=true,
    bookmarksdepth=3   % Uncomment (and tweak) for PDF bookmarks with more levels than the TOC
            ]{hyperref}
%\hyperbaseurl{}   % hyperlinks are relative to this root

\setcounter{tocdepth}{2}  % number chapter, section, subsection

% Tricks for having figures close to where they are defined:
% 1. define less restrictive rules for where to put figures
\setcounter{topnumber}{2}
\setcounter{bottomnumber}{2}
\setcounter{totalnumber}{4}
\renewcommand{\topfraction}{0.85}
\renewcommand{\bottomfraction}{0.85}
\renewcommand{\textfraction}{0.15}
\renewcommand{\floatpagefraction}{0.7}
% 2. ensure all figures are flushed before next section
\usepackage[section]{placeins}
% 3. enable begin{figure}[H] (often leads to ugly pagebreaks)
%\usepackage{float}\restylefloat{figure}


% #ifdef TODONOTES
\usepackage{ifthen,xkeyval,tikz,calc,graphicx}
\usepackage[shadow]{todonotes}
\newcommand{\shortinlinecomment}[3]{%
\todo[size=\normalsize,fancyline,color=orange!40,caption={#3}]{%
 \begin{spacing}{0.75}{\bf #1}: #2\end{spacing}}}
\newcommand{\longinlinecomment}[3]{%
\todo[inline,color=orange!40,caption={#3}]{{\bf #1}: #2}}
% #else
\newcommand{\shortinlinecomment}[3]{{\bf #1}: \emph{#2}}
\newcommand{\longinlinecomment}[3]{{\bf #1}: \emph{#2}}
% #endif

% #ifdef LINENUMBERS
\usepackage[mathlines]{lineno}  % show line numbers
\linenumbers
% #endif

\usepackage[framemethod=TikZ]{mdframed}

% Admonition is an oval gray box
\newmdenv[
  backgroundcolor=gray!5,  %% white with 5%% gray
  skipabove=\topsep,
  skipbelow=\topsep,
  outerlinewidth=0,
  leftmargin=0,
  rightmargin=0,
  roundcorner=5,
]{graybox1mdframed}

\newenvironment{graybox1admon}[1][]{
\begin{graybox1mdframed}[frametitle=#1]
}
{
\end{graybox1mdframed}
}

% #ifdef COLORED_TABLE_ROWS
% color every two table rows
\let\oldtabular\tabular
\let\endoldtabular\endtabular
% #if COLORED_TABLE_ROWS not in ("gray", "blue")
% #define COLORED_TABLE_ROWS gray
% #endif
% #else
% #define COLORED_TABLE_ROWS no
% #endif
% #if COLORED_TABLE_ROWS == "gray"
\definecolor{rowgray}{gray}{0.9}
\renewenvironment{tabular}{\rowcolors{2}{white}{rowgray}%
\oldtabular}{\endoldtabular}
% #elif COLORED_TABLE_ROWS == "blue"
\definecolor{appleblue}{rgb}{0.93,0.95,1.0}  % Apple blue
\renewenvironment{tabular}{\rowcolors{2}{white}{appleblue}%
\oldtabular}{\endoldtabular}
% #endif

% prevent orhpans and widows
\clubpenalty = 10000
\widowpenalty = 10000

% http://www.ctex.org/documents/packages/layout/titlesec.pdf
\usepackage[compact]{titlesec}  % narrower section headings
% #ifdef BLUE_SECTION_HEADINGS
\definecolor{seccolor}{rgb}{0.2,0.2,0.8}
\titleformat{\section}
{\color{seccolor}\normalfont\Large\bfseries}
{\color{seccolor}\thesection}{1em}{}
\titleformat{\subsection}
{\color{seccolor}\normalfont\large\bfseries}
{\color{seccolor}\thesubsection}{1em}{}
% #endif

% --- end of standard preamble for documents ---

\newenvironment{exercise}{}{}
\newcounter{exerno}


% insert custom LaTeX commands...

\makeindex

%-------------------- end preamble ----------------------

\begin{document}

% #endif


% ------------------- main content ----------------------



% ----------------- title -------------------------
% #if LATEX_HEADING == "traditional"
\title{Doconce Quick Reference}

% #elif LATEX_HEADING == "titlepage"

\thispagestyle{empty}
\hbox{\ \ }
\vfill
\begin{center}
{\huge{\bfseries{
\begin{spacing}{1.25}
Doconce Quick Reference
\end{spacing}
}}}

% #elif LATEX_HEADING == "Springer_collection"
\title*{Doconce Quick Reference}
% Short version of title:
%\titlerunning{...}

% #elif LATEX_HEADING == "beamer"
\title{Doconce Quick Reference}
% #else
\begin{center}
{\LARGE\bf
\begin{spacing}{1.25}
Doconce Quick Reference
\end{spacing}
}
\end{center}
% #endif

% ----------------- author(s) -------------------------
% #if LATEX_HEADING == "traditional"
\author{Hans Petter Langtangen\footnote{Center for Biomedical Computing, Simula Research Laboratory and Department of Informatics, University of Oslo.}}

% #elif LATEX_HEADING == "titlepage"
\vspace{1.3cm}

    {\Large\textsf{Hans Petter Langtangen${}^{1, 2}$}}\\ [3mm]
    
\ \\ [2mm]

{\large\textsf{${}^1$Center for Biomedical Computing, Simula Research Laboratory} \\ [1.5mm]}
{\large\textsf{${}^2$Department of Informatics, University of Oslo} \\ [1.5mm]}
% #elif LATEX_HEADING == "Springer_collection"

\author{Hans Petter Langtangen}
% Short version of authors:
%\authorrunning{...}
\institute{Hans Petter Langtangen\at Center for Biomedical Computing, Simula Research Laboratory and Department of Informatics, University of Oslo}

% #elif LATEX_HEADING == "beamer"
\author{Hans Petter Langtangen\inst{1,2}}
\institute{Center for Biomedical Computing, Simula Research Laboratory\inst{1}
\and
Department of Informatics, University of Oslo\inst{2}}
% #else

\begin{center}
{\bf Hans Petter Langtangen${}^{1, 2}$} \\ [0mm]
\end{center}

\begin{center}
% List of all institutions:
\centerline{{\small ${}^1$Center for Biomedical Computing, Simula Research Laboratory}}
\centerline{{\small ${}^2$Department of Informatics, University of Oslo}}
\end{center}
% #endif
% ----------------- end author(s) -------------------------


% #if LATEX_HEADING == "traditional"
\date{Jun 29, 2013}
\maketitle
% #elif LATEX_HEADING == "beamer"
\date{Jun 29, 2013
% <titlepage figure>
}
% #elif LATEX_HEADING == "titlepage"

\ \\ [10mm]
{\large\textsf{Jun 29, 2013}}

\end{center}
\vfill
\clearpage

% #else
\begin{center}
Jun 29, 2013
\end{center}

\vspace{1cm}

% #endif


\tableofcontents
% #ifdef TODONOTES
\listoftodos[List of inline comments]
% #endif

\vspace{1cm} % after toc





\textbf{WARNING: This quick reference is very incomplete!}

\paragraph{Mission.}
Enable writing documentation with much mathematics and
computer code \emph{once, in one place} and include it in traditional {\LaTeX}
books, thesis, and reports, and without extra efforts also make
professionally looking web versions with Sphinx or HTML. Other outlets
include Google's \code{blogger.com}, Wikipedia/Wikibooks, IPython
notebooks, plus a wide variety of formats for documents without
mathematics and code.

\subsection{Supported Formats}

Doconce currently translates files to the following formats:

\begin{itemize}
 \item {\LaTeX} (format \code{latex} and \code{pdflatex})

 \item HTML (format \code{html})

 \item reStructuredText (format \code{rst})

 \item plain (untagged) ASCII (format \code{plain})

 \item Sphinx (format \code{sphinx})

 \item IPython notebook (format \code{ipynb})

 \item MediaWiki (format \code{mwiki})

 \item (Pandoc extended) Markdown (format \code{pandoc})

 \item Googlecode wiki (format \code{gwiki})

 \item Creoloe wiki (format \code{cwiki})

 \item Epydoc (format \code{epydoc})

 \item StructuredText (format \code{st})
\end{itemize}

\noindent
For documents with much code and mathematics, the best (and most supported)
formats are \code{latex}, \code{pdflatex}, \code{sphinx}, and \code{html}; and to a slightly
less extent \code{mwiki} and \code{pandoc}. The HTML format supports blogging on
Google and Wordpress.


\subsection{Emacs syntax support}

The file \href{{https://doconce.googlecode.com/hg/misc/.doconce-mode.el}}{.doconce-mode.el} in the Doconce source distribution
gives a "Doconce Editing Mode" in Emacs. Store the file in the home
directory and add \code{(load-file "~/.doconce-mode.el")} to the \code{.emacs}
file.

Besides syntax highlighting of Doconce documents, this Emacs mode
provides a lot of shortcuts for setting up many elements in a document:


\begin{quote}\begin{tabular}{ll}
\hline
\multicolumn{1}{c}{ Emacs key } & \multicolumn{1}{c}{ Action } \\
\hline
Ctrl+c f                           & figure                             \\
Ctrl+c v                           & movie/video                        \\
Ctrl+c h1                          & heading level 1 (section/h1)       \\
Ctrl+c h2                          & heading level 2 (subsection/h2)    \\
Ctrl+c h3                          & heading level 2 (subsection/h3)    \\
Ctrl+c hp                          & heading for paragraph              \\
Ctrl+c me                          & math environment: !bt equation !et \\
Ctrl+c ma                          & math environment: !bt align !et    \\
Ctrl+c ce                          & code environment: !bc !ec          \\
Ctrl+c cf                          & code from file: @@@CODE            \\
Ctrl+c table2                      & table with 2 columns               \\
Ctrl+c table3                      & table with 3 columns               \\
Ctrl+c table4                      & table with 4 columns               \\
Ctrl+c exer                        & exercise outline                   \\
Ctrl+c slide                       & slide outline                      \\
Ctrl+c help                        & print this table                   \\
\hline
\end{tabular}\end{quote}

\noindent
\subsection{Title, Authors, and Date}

A typical example of giving a title, a set of authors, a date,
and an optional table of contents
reads
\bccq
TITLE: On an Ultimate Markup Language
AUTHOR: H. P. Langtangen at Center for Biomedical Computing, Simula Research Laboratory & Dept. of Informatics, Univ. of Oslo
AUTHOR: Kaare Dump Email: dump@cyb.space.com at Segfault, Cyberspace Inc.
AUTHOR: A. Dummy Author
DATE: today
TOC: on
\eccq
The entire title must appear on a single line.
The author syntax is
\bccq
name Email: somename@adr.net at institution1 & institution2
\eccq
where the email is optional, the "at" keyword is required if one or
more institutions are to be specified, and the \code{&} keyword
separates the institutions (the keyword \code{and} works too).
Each author specification must appear
on a single line.
When more than one author belong to the
same institution, make sure that the institution is specified in an identical
way for each author.

The date can be set as any text different from \code{today} if not the
current date is wanted, e.g., \code{Feb 22, 2016}.

The table of contents is removed by writing \code{TOC: off}.


\subsection{Section Types}
\label{quick:sections}


\begin{quote}\begin{tabular}{ll}
\hline
\multicolumn{1}{c}{ Section type } & \multicolumn{1}{c}{ Syntax } \\
\hline
chapter                                               & \code{========= Heading ========} (9 \code{=})        \\
section                                               & \code{======= Heading =======}    (7 \code{=})        \\
subsection                                            & \code{===== Heading =====}        (5 \code{=})        \\
subsubsection                                         & \code{=== Heading ===}            (3 \code{=})        \\
paragraph                                             & \code{__Heading.__}               (2 \code{_})        \\
abstract                                              & \code{__Abstract.__} Running text...                  \\
appendix                                              & \code{======= Appendix: heading =======} (7 \code{=}) \\
appendix                                              & \code{===== Appendix: heading =====} (5 \code{=})     \\
exercise                                              & \code{======= Exercise: heading =======} (7 \code{=}) \\
exercise                                              & \code{===== Exercise: heading =====} (5 \code{=})     \\
\hline
\end{tabular}\end{quote}

\noindent
Note that abstracts are recognized by starting with \code{__Abstract.__} or
\code{__Summary.__} at the beginning of a line and ending with three or
more \code{=} signs of the next heading.

The \code{Exercise:} keyword kan be substituted by \code{Problem:} or \code{Project:}.
A recommended convention is that an exercise is tied to the text,
a problem can stand on its own, and a project is a comprehensive
problem.

\subsection{Inline Formatting}

Words surrounded by \code{*} are emphasized: \code{*emphasized words*} becomes
\emph{emphasized words}. Similarly, an underscore surrounds words that
appear in boldface: \code{_boldface_} becomes \textbf{boldface}. Colored words
are also possible: the text
\bccq
`color{red}{two red words}`
\eccq
becomes \textcolor{red}{two red words}.

\subsection{Lists}

There are three types of lists: \emph{bullet lists}, where each item starts
with \code{*}, \emph{enumeration lists}, where each item starts with \code{o} and gets
consqutive numbers,
and \emph{description} lists, where each item starts with \code{-} followed
by a keyword and a colon.
\bccq
Here is a bullet list:

 * item1
 * item2
  * subitem1 of item2
  * subitem2 of item2
 * item3

Note that sublists are consistently indented by one or more blanks..
Here is an enumeration list:

 o item1
 o item2
   may appear on
   multiple lines
  o subitem1 of item2
  o subitem2 of item2
 o item3

And finally a description list:

 - keyword1: followed by
   some text
   over multiple
   lines
 - keyword2:
   followed by text on the next line
 - keyword3: and its description may fit on one line
\eccq
The code above follows.

Here is a bullet list:

\begin{itemize}
 \item item1

 \item item2
\begin{itemize}

  \item subitem1 of item2

  \item subitem2 of item2

\end{itemize}

\noindent
 \item item3
\end{itemize}

\noindent
Note that sublists are indented.
Here is an enumeration list:

\begin{enumerate}
\item item1

\item item2
   may appear on
   multiple lines
\begin{enumerate}

 \item subitem1 of item2

 \item subitem2 of item2

\end{enumerate}

\noindent
\item item3
\end{enumerate}

\noindent
And finally a description list:

\begin{description}
 \item[keyword1:] 
   followed by
   some text
   over multiple
   lines

 \item[keyword2:] 
   followed by text on the next line

 \item[keyword3:] 
   and its description may fit on one line
\end{description}

\noindent
\subsection{Comment lines}

Lines starting with \code{#} are treated as comments in the document and
translated to the proper syntax for comments in the output
document. Such comment lines should not appear before {\LaTeX} math
blocks, verbatim code blocks, or lists if the formats \code{rst} and
\code{sphinx} are desired.

Comment lines starting with \code{##} are not propagated to the output
document and can be used for comments that are only interest in
the Doconce file.

Large portions of text can be left out using Preprocess. Just place
\code{# #ifdef EXTRA} and \code{# #endif} around the text. The command line
option \code{-DEXTRA} will bring the text alive again.

When using the Mako preprocessor one can also place comments in
the Doconce source file that will be removed by Mako before
Doconce starts processing the file.


\subsection{Inline comments}

Inline comments meant as messages or notes, to authors during development
in particular,
are enabled by the syntax
\bccq
[name: running text]
\eccq
where \code{name} is the name or ID of an author or reader making the comment,
and \code{running text} is the comment. Here goes an example.
\shortinlinecomment{hpl 1}{ There must be a space after the colon,
but the running text can occupy multiple lines. }{ There must be a }
The inline comments have simple typesetting in most formats, typically boldface
name and everything surrounded by parenthesis, but with {\LaTeX}
output and the \code{-DTOTONOTES} option to \code{ptex2tex} or \code{doconce ptex2tex},
colorful margin or inline boxes (using the \code{todonotes} package)
make it very easy to spot the comments.

Running
\bsys
doconce format html mydoc.do.txt --skip_inline_comments
\esys
removes all inline comments from the output. This feature makes it easy
to turn on and off notes to authors during the development of the document.

All inline comments to readers can also be physically
removed from the Doconce source by
\bsys
doconce remove_inline_comments mydoc.do.txt
\esys

\subsection{Verbatim/Computer Code}

Inline verbatim code is typeset within back-ticks, as in
\bccq
Some sentence with `words in verbatim style`.
\eccq
resulting in Some sentence with \code{words in verbatim style}.

Multi-line blocks of verbatim text, typically computer code, is typeset
in between \code{!bc xxx} and \code{!ec} directives, which must appear on the
beginning of the line. A specification \code{xxx} indicates what verbatim
formatting style that is to be used. Typical values for \code{xxx} are
nothing, \code{cod} for a code snippet, \code{pro} for a complete program,
\code{sys} for a terminal session, \code{dat} for a data file (or output from a
program),
\code{Xpro} or \code{Xcod} for a program or code snipped, respectively,
in programming \code{X}, where \code{X} may be \code{py} for Python,
\code{cy} for Cython, \code{sh} for Bash or other Unix shells,
\code{f} for Fortran, \code{c} for C, \code{cpp} for C++, \code{m} for MATLAB,
\code{pl} for Perl. For output in \code{latex} one can let \code{xxx} reflect any
defined verbatim environment in the \code{ptex2tex} configuration file
(\code{.ptex2tex.cfg}). For \code{sphinx} output one can insert a comment
\bccq
# sphinx code-blocks: pycod=python cod=fortran cppcod=c++ sys=console
\eccq
that maps environments (\code{xxx}) onto valid language types for
Pygments (which is what \code{sphinx} applies to typeset computer code).

The \code{xxx} specifier has only effect for \code{latex} and
\code{sphinx} output. All other formats use a fixed monospace font for all
kinds of verbatim output.

Here is an example of computer code (see the source of this document
for exact syntax):

\bcod
from numpy import sin, cos, exp, pi

def f(x, y, z, t):
    return exp(-t)*sin(pi*x)*sin(pi*y)*cos(2*pi*z)
\ecod

% When showing copy from file in !bc envir, indent a character - otherwise
% ptex2tex is confused and starts copying...
Computer code can also be copied from a file:
\bccq
 @@@CODE doconce_program.sh
 @@@CODE doconce_program.sh  fromto: doconce clean@^doconce split_rst
 @@@CODE doconce_program.sh  from-to: doconce clean@^doconce split_rst
\eccq
The \code{@@@CODE} identifier must appear at the very beginning of the line.
The first specification copies the complete file \code{doconce_program.sh}.
The second specification copies from the first line matching the \emph{regular
expression} \code{doconce clean} up to, but not including the line
matching the \emph{regular expression} \code{^doconce split_rst}.
The third specification behaves as the second, but the line matching
the first regular expression is not copied (aimed at copying
text between begin-end comment pair in the file).

The copied line from file are in this example put inside \code{!bc shpro}
and \code{!ec} directives, if a complete file is copied, while the
directives become \code{!bc shcod} and \code{!ec} when a code snippet is copied
from file. In general, for a filename extension \code{.X}, the environment
becomes \code{!bc Xpro} or \code{!bc Xcod} for a complete program or snippet,
respectively. The enivorments (\code{Xcod} and \code{Xpro}) are only active
for \code{latex} and \code{sphinx} outout.

Important warnings:

\begin{itemize}
 \item A code block must come after some plain sentence (at least for successful
   output in reStructredText), not directly after a section/paragraph heading,
   table, comment, figure, or movie.

 \item Verbatim code blocks inside lists can be ugly typeset in some
   output formats. A more robust approach is to replace the list by
   paragraphs with headings.
\end{itemize}

\noindent
\subsection{{\LaTeX} Mathematics}

Doconce supports inline mathematics and blocks of mathematics, using
standard {\LaTeX} syntax. The output formats \code{html}, \code{sphinx}, \code{latex},
pdflatex`, \code{pandoc}, and \code{mwiki} work with this syntax while all other
formats will just display the raw {\LaTeX} code.

Inline expressions are written in the standard
{\LaTeX} way with the mathematics surrounded by dollar signs, as in
$Ax=b$. To help increase readability in other formats than \code{sphinx},
\code{latex}, and \code{pdflatex}, inline mathematics may have a more human
readable companion expression. The syntax is like
\bccq
$\sin(\norm{\bf u})$|$sin(||u||)$
\eccq
That is, the {\LaTeX} expression appears to the left of a vertical bar (pipe
symbol) and the more readable expression appears to the right. Both
expressions are surrounded by dollar signs.

Blocks of {\LaTeX} mathematics are written within
\code{!bt}
and
\code{!et} (begin/end TeX) directives starting on the beginning of a line:

\bccq
!bt
\begin{align*}
\nabla\cdot \pmb{u} &= 0,\\ 
\nabla\times \pmb{u} &= 0.
\end{align*}
!et
\eccq

This {\LaTeX} code gets rendered as

\begin{align*}
\nabla\cdot \pmb{u} &= 0,\\ 
\nabla\times \pmb{u} &= 0.
\end{align*}
Here is a single equation:

\bccq
!bt
\[ \frac{\partial\pmb{u}}{\partial t} + \pmb{u}\cdot\nabla\pmb{u} = 0.\]
!et
\eccq
which results in

\[ \frac{\partial\pmb{u}}{\partial t} + \pmb{u}\cdot\nabla\pmb{u} = 0.\]

Any {\LaTeX} syntax is accepted, but if output in the \code{sphinx}, \code{pandoc},
\code{mwiki}, \code{html}, or \code{ipynb} formats
is also important, one should follow these rules:

\begin{itemize}
  \item Use only the equation environments \code{\[}, \code{\]},
    \code{equation}, \code{equation*}, \code{align}, and \code{align*}.

  \item MediaWiki (\code{mwiki}) does not support references to equations.
\end{itemize}

\noindent
(Doconce performs extensions to \code{sphinx} and other formats such that
labels in \code{align} environments work well.)


\begin{graybox1admon}[Notice.]
{\LaTeX} supports lots of fancy formatting, for example, multiple
plots in the same figure, footnotes, margin notes, etc.
Allowing other output formats, such as \code{sphinx}, makes it necessary
to only utilze very standard {\LaTeX} and avoid, for instance, more than
one plot per figure. However, one can use preprocessor if-tests on
the format (typically \code{if FORMAT in ("latex", "pdflatex")}) to
include special code for \code{latex} and \code{pdflatex} output and more
straightforward typesetting for other formats. In this way, one can
also allow advanced {\LaTeX} features and fine tuning of resulting
PDF document.
\end{graybox1admon}

\paragraph{LaTeX Newcommands.}
The author can define \code{newcommand} statements in files with names
\code{newcommands*.tex}. Such commands should only be used for mathematics
(other {\LaTeX} constructions are only understood by {\LaTeX} itself).
The convention is that \code{newcommands_keep.tex}
contains the newcommands that are kept in the document, while
those in \code{newcommands_replace.tex} will be replaced by their full
{\LaTeX} code. This conventions helps make readable documents in formats
without {\LaTeX} support. For \code{html}, \code{sphinx}, \code{latex}, \code{pdflatex},
\code{mwiki}, \code{ipynb}, and \code{pandoc}, the mathematics in newcommands is
rendered nicely anyway.


\subsection{Hyperlinks}

Links use either a link text or the raw URL:

\bccq
Here is some "some link text": "http://some.net/address"
(as in "search google": "http://google.com")
or just the raw address: URL: "http://google.com".

Links to files typeset in verbatim mode applies backtics:
"`myfile.py`": "http://some.net/some/place/myfile.py".

Mail addresses works too: send problems to
"`hpl@simula.no`": "mailto:hpl@simula.no"
or just "send mail": "mailto:hpl@simula.no".
\eccq

\subsection{Figures and Movies}

Figures and movies have almost equal syntax:
\bccq
FIGURE: [relative/path/to/figurefile, width=500 frac=0.8] Here goes the caption which must be on a single line. label{some:fig:label}

MOVIE: [relative/path/to/moviefile, width=500] Here goes the caption which must be on a single line. label{some:fig:label}

\eccq
Note three important syntax details:

\begin{enumerate}
 \item A mandatory comma after the figure/movie filename,

 \item no comments between \code{width}, \code{height}, and \code{frac} and no spaces
    around the \code{=} characters,

 \item all of the command must appear on a single line,

 \item there must be a blank line after the command.
\end{enumerate}

\noindent
The figure file can be listed without extension. Doconce will then find
the version of the file with the most appropriate extension for the chosen
output format. If not suitable version is found, Doconce will convert
another format to the needed one.

The caption is optional. If omitted, the figure will be inlined (meaning
no use of any figure environment in HTML or {\LaTeX} formats). The \code{width}
and \code{height} parameters affect HTML formats (\code{html}, \code{rst}, \code{sphinx}),
while \code{frac} is the width of the image as a fraction of the total text
width in the \code{latex} and \code{pdflatex} formats.

Movie files can either be a video or a wildcard expression for a
series of frames. In the latter case, a simple device in an HTML page
will display the individual frame files as a movie.

Combining several image files into one can be done by the
\bsys
doconce combine_images image1 image2 ... output_image
\esys
This command applies \code{montage} or PDF-based tools to combine the images
to get the highest quality.

YouTube and Vimeo movies will be embedded in \code{html} and \code{sphinx} documents
and otherwise be represented by a link. The syntax is

\bccq
MOVIE: [http://www.youtube.com/watch?v=_O7iUiftbKU, width=420 height=315] YouTube movie.

MOVIE: [http://vimeo.com/55562330, width=500 height=278] Vimeo movie.

\eccq
The latter results in

 Vimeo movie. \href{{http://vimeo.com/55562330}}{\nolinkurl{http://vimeo.com/55562330}}


\subsection{Tables}

The table in Section~\ref{quick:sections} was written with this
syntax:
\bccq
|----------------c--------|------------------c--------------------|
|      Section type       |        Syntax                         |
|----------------l--------|------------------l--------------------|
| chapter                 | `========= Heading ========` (9 `=`)  |
| section                 | `======= Heading =======`    (7 `=`)  |
| subsection              | `===== Heading =====`        (5 `=`)  |
| subsubsection           | `=== Heading ===`            (3 `=`)  |
| paragraph               | `__Heading.__`               (2 `_`)  |
|-----------------------------------------------------------------|
\eccq

Note that

\begin{itemize}
 \item Each line begins and ends with a vertical bar (pipe symbol).

 \item Column data are separated by a vertical bar (pipe symbol).

 \item There may be horizontal rules, i.e., lines with dashes for
   indicating the heading and the end of the table, and these may
   contain characters 'c', 'l', or 'r' for how to align headings or
   columns. The first horizontal rule may indicate how to align
   headings (center, left, right), and the horizontal rule after the
   heading line may indicate how to align the data in the columns
   (center, left, right).

 \item If the horizontal rules are without alignment information there should
   be no vertical bar (pipe symbol) between the columns. Otherwise, such
   a bar indicates a vertical bar between columns in {\LaTeX}.

 \item Many output formats are so primitive that heading and column alignment
   have no effect.
\end{itemize}

\noindent
The command-line option \code{--tables2csv} (to \code{doconce format})
makes Doconce dump each table to CSV format in a file \code{table_X.csv},
where \code{X} is the table number. This feature makes it easy to
load tables into spreadsheet programs for further analysis.

\subsection{Labels and References}

The notion of labels and references (as well as bibliography and index)
is adopted
from {\LaTeX} with a very similar syntax. As in {\LaTeX}, a label can be
inserted anywhere, using the syntax
\bccq
label{name}
\eccq
with no backslash
preceding the label keyword. It is common practice to choose \code{name}
as some hierarchical name, say \code{a:b:c}, where \code{a} and \code{b} indicate
some abbreviations for a section and/or subsection for the topic and
\code{c} is some name for the particular unit that has a label.

A reference to the label \code{name} is written as
\bccq
ref{name}
\eccq
again with no backslash before \code{ref}.

Use labels for sections and equations only, and preceed the reference
by "Section" or "Chapter", or in case of an equation, surround the
reference by parenthesis.


\subsection{Citations and Bibliography}

Single citations are written as
\bccq
cite{name}
\eccq
where \code{name} is a logical name
of the reference (again, {\LaTeX} writers must not insert a backslash).
Bibliography citations often have \code{name} on the form
\code{Author1_Author2_YYYY}, \code{Author_YYYY}, or \code{Author1_etal_YYYY}, where
\code{YYYY} is the year of the publication.
Multiple citations at once is possible by separating the logical names
by comma:
\bccq
cite{name1,name2,name3}
\eccq

The bibliography is specified by a line \code{BIBFILE: papers.pub},
where \code{papers.pub} is a publication database in the
\href{{https://bitbucket.org/logg/publish}}{Publish} format.
\textsc{Bib}\negthinspace{\TeX} \code{.bib} files can easily be combined to a Publish database
(which Doconce needs to create bibliographies in other formats
than {\LaTeX}).

\subsection{Generalized Citations}

There is a \emph{generalized referencing} feature in Doconce that allows
a reference with \code{ref} to have one formulation if the label is
in the same document and another formulation if the reference is
to an item in an external document. This construction makes it easy
to work with many small, independent documents in parallel with
a book assembly of some of the small elements.
The syntax of a generalized reference is
\bccq
ref[internal][cite][external]

# Example:
As explained in
ref[Section ref{subsec:ex}][in cite{testdoc:12}][a "section":
"testdoc.html#___sec2" in the document
"A Document for Testing Doconce": "testdoc.html" cite{testdoc:12}],
Doconce documents may include movies.
\eccq
The output from a generalized reference is the text \code{internal} if all
\code{ref{label}} references in \code{internal} are references to labels in the
present document. Otherwise, if cite is non-empty and the format is
\code{latex} or \code{pdflatex} one assumes that the references in \code{internal}
are to external documents declared by a comment line \code{#
Externaldocuments: testdoc, mydoc} (usually after the title, authors,
and date). In this case the output text is \code{internal cite} and the
{\LaTeX} package \code{xr} is used to handle the labels in the external
documents.  If none of the two situations above applies, the
\code{external} text will be the output.

\subsection{Index of Keywords}

Doconce supports creating an index of keywords. A certain keyword
is registered for the index by a syntax like (no
backslash!)
\bccq
index{name}
\eccq
It is recommended to place any index of this type outside
running text, i.e., after (sub)section titles and in the space between
paragraphs. Index specifications placed right before paragraphs also
gives the doconce source code an indication of the content in the
forthcoming text. The index is only produced for the \code{latex},
\code{pdflatex}, \code{rst}, and \code{sphinx} formats.

\subsection{Capabilities of The Program \protect\code{doconce} }

The \code{doconce} program can be used for a number of purposes besides
transforming a \code{.do.txt} file to some format. Here is the
list of capabilities:

\bshpro
Usage: doconce command [optional arguments]
commands: format help sphinx_dir subst replace replace_from_file clean spellcheck ptex2tex expand_commands combine_images guess_encoding change_encoding gwiki_figsubst md2html remove_inline_comments grab remove remove_exercise_answers split_rst split_html slides_html slides_beamer latin2html latex_header latex_footer bbl2rst html_colorbullets list_labels teamod sphinxfix_localURLs make_figure_code_links latex_exercise_toc insertdocstr old2new_format latex2doconce latex_dislikes pygmentize makefile diff gitdiff fix_bibtex4publish csv2table


# transform doconce file to another format
doconce format html|latex|pdflatex|rst|sphinx|plain|gwiki|mwiki|cwiki|pandoc|st|epytext file.do.txt

# substitute a phrase by another using regular expressions
doconce subst [-s -m -x --restore] regex-pattern regex-replacement file1 file2 ...
(-s is the re.DOTALL modifier, -m is the re.MULTILINE modifier,
 -x is the re.VERBOSE modifier, --restore copies backup files back again)

# replace a phrase by another literally
doconce replace from-text to-text file1 file2 ...
(exact text substutition)

# doconce replace using from and to phrases from file
doconce replace_from_file file-with-from-to file1 file2 ...
(exact text substitution, but a set of from-to relations)

# gwiki format requires substitution of figure file names by URLs
doconce gwiki_figsubst file.gwiki URL-of-fig-dir

# remove all inline comments in a doconce file
doconce remove_inline_comments file.do.txt

# create a directory for the sphinx format
doconce sphinx_dir author='John Doe' title='Long title' \
    short_title="Short title" version=0.1 \
    dirname=sphinx-rootdir theme=default logo=mylogo.png \
    do_file [do_file2 do_file3 ...]
(requires sphinx version >= 1.1)

# replace latex-1 (non-ascii) characters by html codes
doconce latin2html file.html

# walk through a directory tree and insert doconce files as
# docstrings in *.p.py files
doconce insertdocstr rootdir

# remove all files that the doconce format can regenerate
doconce clean

# print the header (preamble) for latex file
doconce latex_header

# print the footer for latex files
doconce latex_footer

# change encoding
doconce change_encoding utf-8 latin1 filename

# guess the encoding in a text
doconce guess_encoding filename

# transform a .bbl file to a .rst file with reST bibliography format
doconce bbl2rst file.bbl

# split a sphinx/rst file into parts
doconce format sphinx complete_file
doconce split_rst complete_file        # !split delimiters
doconce sphinx_dir complete_file
python automake_sphinx.py

# edit URLs to local files and place them in _static
doconce sphinxfix_local_URLs file.rst

# split an html file into parts according to !split commands
doconce split_html complete_file.html

# create HTML slides from a (doconce) html file
doconce slides_html slide_type complete_file.html

# create LaTeX Beamer slides from a (doconce) latex/pdflatex file
doconce slides_beamer complete_file.tex

# replace bullets in lists by colored bullets
doconce html_colorbullets file1.html file2.html ...

# grab selected text from a file
doconce grab   --from[-] from-text [--to[-] to-text] somefile

# remove selected text from a file
doconce remove --from[-] from-text [--to[-] to-text] somefile

# run spellcheck on a set of files
doconce spellcheck [-d .mydict.txt] *.do.txt

# transform ptex2tex files (.p.tex) to ordinary latex file
# and manage the code environments
doconce ptex2tex mydoc -DMINTED pycod=minted sys=Verbatim \
        dat=\begin{quote}\begin{verbatim};\end{verbatim}\end{quote}

# make HTML file via pandoc from Markdown (.md) file
doconce md2html file

# make LaTeX file via pandoc from Markdown (.md) file
doconce md2latex file

# expand short cut commands to full form in files
doconce expand_commands file1 file2 ...

# combine several images into one
doconce combine_images image1 image2 ... output_file

# insert a table of exercises in a latex file myfile.p.tex
doconce latex_exercise_toc myfile

# list all labels in a document (for purposes of cleaning them up)
doconce list_labels myfile

# translate a latex document to doconce (requires usually manual fixing)
doconce latex2doconce latexfile

# check if there are problems with translating latex to doconce
doconce latex_dislikes latexfile

# typeset a doconce document with pygments (for pretty print of doconce itself)
doconce pygmentize myfile [pygments-style]

# generate a make.sh script for translating a doconce file to various formats
doconce makefile docname doconcefile [html sphinx pdflatex ...]

# fix common problems in bibtex files for publish import
doconce fix_bibtex4publish file1.bib file2.bib ...

# find differences between two files
doconce diff file1.do.txt file2.do.txt [diffprog]
(diffprog can be difflib, diff, pdiff, latexdiff, kdiff3, diffuse, ...)

# find differences between the last two Git versions of several files
doconce gitdiff file1 file2 file3 ...

# convert csv file to doconce table format
doconce csv2table somefile.csv
\eshpro

\subsection{Exercises}

Doconce supports \emph{Exercise}, \emph{Problem}, \emph{Project}, and \emph{Example}.
These are typeset
as ordinary sections and referred to by their section labels.
Exercise, problem, project, or example sections contains certain \emph{elements}:

\begin{itemize}
  \item a headline at the level of a subsection
    containing one of the words "Exercise:", "Problem:",
    "Project:", or "Example:", followed by a title (required)

  \item a label (optional)

  \item a solution file (optional)

  \item name of file with a student solution (optional)

  \item main exercise text (required)

  \item a short answer (optional)

  \item a full solution (optional)

  \item one or more hints (optional)

  \item one or more subexercises (subproblems, subprojects), which can also
    contain a text, a short answer, a full solution, name student file
    to be handed in, and one or more hints (optional)
\end{itemize}

\noindent
A typical sketch of a a problem without subexercises goes as follows:
\bccq
===== Problem: Derive the Formula for the Area of an Ellipse =====
label{problem:ellipsearea1}
file=ellipse_area.pdf
solution=ellipse_area1_sol.pdf

Derive an expression for the area of an ellipse by integrating
the area under a curve that defines half of the allipse.
Show each step in the mathematical derivation.

!bhint
Wikipedia has the formula for the curve.
!ehint

!bhint
"Wolframalpha": "http://wolframalpha.com" can perhaps
compute the integral.
!ehint
\eccq
If the exercise type (Exercise, Problem, Project, or Example)
is enclosed in braces, the type is left out of the title in the
output. For example, the if the title line above reads

\bccq
===== {Problem}: Derive the Formula for the Area of an Ellipse =====
\eccq
the title becomes just "Derive the ...".

An exercise with subproblems, answers and full solutions has this
setup-up:

\bccq
===== Exercise: Determine the Distance to the Moon =====
label{exer:moondist}

Intro to this exercise. Questions are in subexercises below.

!bsubex
Subexercises are numbered a), b), etc.

file=subexer_a.pdf

!bans
Short answer to subexercise a).
!eans

!bhint
First hint to subexercise a).
!ehint

!bhint
Second hint to subexercise a).
!ehint
!esubex

!bsubex
Here goes the text for subexercise b).

file=subexer_b.pdf

!bhint
A hint for this subexercise.
!ehint

!bsol
Here goes the solution of this subexercise.
!esol
!esubex

!bremarks
At the very end of the exercise it may be appropriate to summarize
and give some perspectives. The text inside the !bremarks-!eremarks
directives is always typeset at the end of the exercise.
!eremarks

!bsol
Here goes a full solution of the whole exercise.
!esol

\eccq
By default, answers, solutions, and hints are typeset as paragraphs.
The command-line arguments \code{--without_answers} and \code{--without_solutions}
turn off output of answers and solutions, respectively, except for
examples.


\subsection{Environments}

Doconce environments start with \code{!benvirname} and end with \code{!eenvirname},
where \code{envirname} is the name of the environment. Here is a listing of
the environments:

\begin{itemize}
 \item \code{c}: computer code (or verbatim text)

 \item \code{t}: math blocks with {\LaTeX} syntax

 \item \code{subex}: sub-exercise

 \item \code{ans}: short answer to exercise or sub-exercise

 \item \code{sol}: full solution to exercise or sub-exercise

 \item \code{quote}: indented text

 \item \code{notice}, \code{summary}, \code{warning}, \code{question}, \code{hint}: admonition boxes with
    custom title, special icon, and (frequently) background color

 \item \code{pop}: text to gradually pop up in slide presentations

 \item \code{slidecell}: indication of cells in a grid layout for elements on a
   slide
\end{itemize}

\noindent
\subsection{Preprocessing}

Doconce documents may utilize a preprocessor, either \code{preprocess} and/or
\code{mako}. The former is a C-style preprocessor that allows if-tests
and including other files (but not macros with arguments).
The \code{mako} preprocessor is much more advanced - it is actually a full
programming language, very similar to Python.

The command \code{doconce format} first runs \code{preprocess} and then \code{mako}.
Here is a typical example on utilizing \code{preprocess} to include another
document, "comment out" a large portion of text, and to write format-specific
constructions:

\bccq
# #include "myotherdoc.do.txt"

# #if FORMAT in ("latex", "pdflatex")
\begin{table}
\caption{Some words... label{mytab}}
\begin{tabular}{lrr}
\hline\noalign{\smallskip}
\multicolumn{1}{c}{time} & \multicolumn{1}{c}{velocity} & \multicolumn{1}{c}{acceleration} \\ 
\hline
0.0          & 1.4186       & -5.01        \\ 
2.0          & 1.376512     & 11.919       \\ 
4.0          & 1.1E+1       & 14.717624    \\ 
\hline
\end{tabular}
\end{table}
# #else
  |--------------------------------|
  |time  | velocity | acceleration |
  |--l--------r-----------r--------|
  | 0.0  | 1.4186   | -5.01        |
  | 2.0  | 1.376512 | 11.919       |
  | 4.0  | 1.1E+1   | 14.717624    |
  |--------------------------------|
# #endif

# #ifdef EXTRA_MATERIAL
....large portions of text...
# #endif
\eccq

With the \code{mako} preprocessor the if-else tests have slightly different syntax.
An \href{{http://hplgit.github.com/bioinf-py/}}{example document} contains
some illustrations on how to utilize \code{mako} (clone the GitHub project and
examine the Doconce source and the \code{doc/src/make.sh} script).

\subsection{Resources}

\begin{itemize}
 \item Excellent "Sphinx Tutorial" by C. Reller: "http://people.ee.ethz.ch/~creller/web/tricks/reST.html"
\end{itemize}

\noindent

% ------------------- end of main content ---------------


% #ifdef PREAMBLE
\printindex

\end{document}
% #endif

