%%
%% Automatically generated file from Doconce source
%% (https://github.com/hplgit/doconce/)
%%
% #ifdef PTEX2TEX_EXPLANATION
%%
%% The file follows the ptex2tex extended LaTeX format, see
%% ptex2tex: http://code.google.com/p/ptex2tex/
%%
%% Run
%%      ptex2tex myfile
%% or
%%      doconce ptex2tex myfile
%%
%% to turn myfile.p.tex into an ordinary LaTeX file myfile.tex.
%% (The ptex2tex program: http://code.google.com/p/ptex2tex)
%% Many preprocess options can be added to ptex2tex or doconce ptex2tex
%%
%%      ptex2tex -DMINTED -DPALATINO -DA6PAPER -DLATEX_HEADING=traditional myfile
%%      doconce ptex2tex myfile -DMINTED -DLATEX_HEADING=titlepage
%%
%% ptex2tex will typeset code environments according to a global or local
%% .ptex2tex.cfg configure file. doconce ptex2tex will typeset code
%% according to options on the command line (just type doconce ptex2tex to
%% see examples).
% #endif

% #ifndef LATEX_HEADING
% #define LATEX_HEADING "doconce_heading"
% #endif

% #ifndef PREAMBLE
% #if LATEX_HEADING == "Springer_collection"
% #undef PREAMBLE
% #else
% #define PREAMBLE
% #endif
% #endif


% #ifdef PREAMBLE
%-------------------- begin preamble ----------------------

\documentclass[%
oneside,                 % oneside: electronic viewing, twoside: printing
final,                   % or draft (marks overfull hboxes)
10pt]{article}

\listfiles               % print all files needed to compile this document

% #ifdef A4PAPER
\usepackage[a4paper]{geometry}
% #endif
% #ifdef A6PAPER
% a6paper is suitable for mobile devices
\usepackage[%
  a6paper,
  text={90mm,130mm},
  inner={5mm},           % inner margin (two sided documents)
  top=5mm,
  headsep=4mm
  ]{geometry}
% #endif

\usepackage{relsize,epsfig,makeidx,color,setspace,amsmath,amsfonts}
\usepackage[table]{xcolor}
\usepackage{bm,microtype}
\usepackage{ptex2tex}

% #ifdef MINTED
\usepackage{minted}
\usemintedstyle{default}
% #endif

% #ifdef XELATEX
% xelatex settings
\usepackage{fontspec}
\usepackage{xunicode}
\defaultfontfeatures{Mapping=tex-text} % To support LaTeX quoting style
\defaultfontfeatures{Ligatures=TeX}
\setromanfont{Kinnari}
% Examples of font types (Ubuntu): Gentium Book Basic (Palatino-like),
% Liberation Sans (Helvetica-like), Norasi, Purisa (handwriting), UnDoum
% #else
%\usepackage[latin1]{inputenc}
\usepackage[utf8]{inputenc}
% #ifdef HELVETICA
% Set helvetica as the default font family:
\RequirePackage{helvet}
\renewcommand\familydefault{phv}
% #endif
% #ifdef PALATINO
% Set palatino as the default font family:
\usepackage[sc]{mathpazo}    % Palatino fonts
\linespread{1.05}            % Palatino needs extra line spread to look nice
% #endif
% #endif

% Hyperlinks in PDF:
\definecolor{linkcolor}{rgb}{0,0,0.4}
\usepackage[%
    colorlinks=true,
    linkcolor=linkcolor,
    urlcolor=linkcolor,
    citecolor=black,
    filecolor=black,
    %filecolor=blue,
    pdfmenubar=true,
    pdftoolbar=true,
    bookmarksdepth=3   % Uncomment (and tweak) for PDF bookmarks with more levels than the TOC
            ]{hyperref}
%\hyperbaseurl{}   % hyperlinks are relative to this root

\setcounter{tocdepth}{2}  % number chapter, section, subsection

% #ifdef COLORED_TABLE_ROWS
% color every two table rows
\let\oldtabular\tabular
\let\endoldtabular\endtabular
% #if COLORED_TABLE_ROWS not in ("gray", "blue")
% #define COLORED_TABLE_ROWS gray
% #endif
% #else
% #define COLORED_TABLE_ROWS no
% #endif
% #if COLORED_TABLE_ROWS == "gray"
\definecolor{rowgray}{gray}{0.9}
\renewenvironment{tabular}{\rowcolors{2}{white}{rowgray}%
\oldtabular}{\endoldtabular}
% #elif COLORED_TABLE_ROWS == "blue"
\definecolor{appleblue}{rgb}{0.93,0.95,1.0}  % Apple blue
\renewenvironment{tabular}{\rowcolors{2}{white}{appleblue}%
\oldtabular}{\endoldtabular}
% #endif

% prevent orhpans and widows
\clubpenalty = 10000
\widowpenalty = 10000

% http://www.ctex.org/documents/packages/layout/titlesec.pdf
\usepackage[compact]{titlesec}  % narrower section headings
% #ifdef BLUE_SECTION_HEADINGS
\definecolor{seccolor}{rgb}{0.2,0.2,0.8}
\titleformat{\section}
{\color{seccolor}\normalfont\Large\bfseries}
{\color{seccolor}\thesection}{1em}{}
\titleformat{\subsection}
{\color{seccolor}\normalfont\large\bfseries}
{\color{seccolor}\thesubsection}{1em}{}
% #endif

% --- end of standard preamble for documents ---


% insert custom LaTeX commands...

\makeindex

%-------------------- end preamble ----------------------

\begin{document}

% #endif

\renewcommand{\u}{\pmb{u}}

\newcommand{\xbm}{\bm{x}}
\newcommand{\normalvecbm}{\bm{n}}
\newcommand{\ubm}{\bm{u}}


\newcommand{\x}{\pmb{x}}
\newcommand{\normalvec}{\pmb{n}}
\newcommand{\Ddt}[1]{\frac{D#1}{dt}}
\newcommand{\halfi}{1/2}
\newcommand{\half}{\frac{1}{2}}
\newcommand{\report}{test report}

% ------------------- main content ----------------------



% ----------------- title -------------------------
% #if LATEX_HEADING == "traditional"
\title{How various formats can deal with {\LaTeX} math}

% #elif LATEX_HEADING == "titlepage"

\thispagestyle{empty}
\hbox{\ \ }
\vfill
\begin{center}
{\huge{\bfseries{
\begin{spacing}{1.25}
How various formats can deal with {\LaTeX} math
\end{spacing}
}}}

% #elif LATEX_HEADING == "Springer_collection"
\title*{How various formats can deal with {\LaTeX} math}
% Short version of title:
%\titlerunning{...}

% #elif LATEX_HEADING == "beamer"
\title{How various formats can deal with {\LaTeX} math}
% #else
\begin{center}
{\LARGE\bf
\begin{spacing}{1.25}
How various formats can deal with {\LaTeX} math
\end{spacing}
}
\end{center}
% #endif

% ----------------- author(s) -------------------------
% #if LATEX_HEADING == "traditional"
\author{HPL}

% #elif LATEX_HEADING == "titlepage"
\vspace{1.3cm}

    {\Large\textsf{HPL${}^{}$}}\\ [3mm]
    
\ \\ [2mm]

% #elif LATEX_HEADING == "Springer_collection"

\author{HPL}
% Short version of authors:
%\authorrunning{...}
\institute{HPL}

% #elif LATEX_HEADING == "beamer"
\author{HPL\inst{}}
\institute{}
% #else

\begin{center}
{\bf HPL${}^{}$} \\ [0mm]
\end{center}

\begin{center}
% List of all institutions:
\end{center}
% #endif
% ----------------- end author(s) -------------------------


% #if LATEX_HEADING == "traditional"
\date{Aug 16, 2013}
\maketitle
% #elif LATEX_HEADING == "beamer"
\date{Aug 16, 2013
% <titlepage figure>
}
% #elif LATEX_HEADING == "titlepage"

\ \\ [10mm]
{\large\textsf{Aug 16, 2013}}

\end{center}
\vfill
\clearpage

% #else
\begin{center}
Aug 16, 2013
\end{center}

\vspace{1cm}

% #endif


This document is translated to the format \textbf{pdflatex}. The purpose is to
test math and doconce and various output formats.

\paragraph{Test 1: Inline math.}
Here is a sentence contains the equation $u(t)=e^{-at}$.

\paragraph{Test 2: A single equation without label.}
Here it is

\[ u(t)=e^{-at} \]

\paragraph{Test 3: A single equation with label.}
Here it is as a one-line
latex code,

\bccq
!bt
\begin{equation} u(t)=e^{-at} label{eq1}\end{equation}
!et
\eccq
looking like

\begin{equation} u(t)=e^{-at} \label{eq1}\end{equation}
and as a three-line latex code:

\bccq
!bt
\begin{equation}
u(t)=e^{-at} label{eq1b}
\end{equation}
!et
\eccq
looking like

\begin{equation}
u(t)=e^{-at} \label{eq1b}
\end{equation}
This equation has label (\ref{eq1b}).


\paragraph{Test 4: Multiple, aligned equations without label.}
Only the align
environment is supported by other formats than {\LaTeX} for typesetting
multiple, aligned equations. The code reads

\bccq
!bt
\begin{align*}
u(t)&=e^{-at}\\ 
v(t) - 1 &= \frac{du}{dt}
\end{align*}
!et
\eccq
and results in

\begin{align*}
u(t)&=e^{-at}\\ 
v(t) - 1 &= \frac{du}{dt}
\end{align*}

\paragraph{Test 5: Multiple, aligned equations with label.}
We use align with
labels:

\bccq
!bt
\begin{align}
u(t)&=e^{-at}
label{eq2b}\\ 
v(t) - 1 &= \frac{du}{dt}
label{eq3b}
\end{align}
!et
\eccq
and results in

\begin{align}
u(t)&=e^{-at} \label{eq2b}\\ 
v(t) - 1 &= \frac{du}{dt} \label{eq3b}
\end{align}
We can refer to the last equations as the system (\ref{eq2b})-(\ref{eq3b}).




\paragraph{Test 6: Multiple, aligned eqnarray equations without label.}
Let us
try the old eqnarray environment.

\bccq
!bt
\begin{eqnarray*}
u(t)&=& e^{-at}\\ 
v(t) - 1 &=& \frac{du}{dt}
\end{eqnarray*}
!et
\eccq
and results in

\begin{eqnarray*}
u(t)&=& e^{-at}\\ 
v(t) - 1 &=& \frac{du}{dt}
\end{eqnarray*}

\paragraph{Test 7: Multiple, eqnarrayed equations with label.}
We use eqnarray with
labels:

\bccq
!bt
\begin{eqnarray}
u(t)&=& e^{-at}
label{eq2c}\\ 
v(t) - 1 &=& \frac{du}{dt}
label{eq3c}
\end{eqnarray}
!et
\eccq
and results in

\begin{eqnarray}
u(t)&=& e^{-at} \label{eq2c}\\ 
v(t) - 1 &=& \frac{du}{dt} \label{eq3c}
\end{eqnarray}
Can we refer to the last equations as the system (\ref{eq2c})-(\ref{eq3c})?

\paragraph{Test 8: newcommands and boldface bm vs pmb.}
We have

\[ \color{blue}{\frac{\partial\u}{\partial t}} +
\nabla\cdot\nabla\u = \nu\nabla^2\u -
\frac{1}{\varrho}\nabla p,\]
and $\nabla\u (\x)\cdot\normalvec$
with plain old pmb. Here are the same formulas using \code{\bm}:

\[ \color{blue}{\frac{\partial\ubm}{\partial t}} +
\nabla\cdot\nabla\ubm = \nu\nabla^2\ubm -
\frac{1}{\varrho}\nabla p,\]
and $\nabla\ubm (\xbm)\cdot\normalvecbm$.


% ------------------- end of main content ---------------


% #ifdef PREAMBLE
\printindex

\end{document}
% #endif

