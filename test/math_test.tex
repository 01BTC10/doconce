%%
%% Automatically generated file from Doconce source
%% (https://github.com/hplgit/doconce/)
%%




%-------------------- begin preamble ----------------------

\documentclass[%
oneside,                 % oneside: electronic viewing, twoside: printing
final,                   % or draft (marks overfull hboxes)
10pt]{article}

\listfiles               % print all files needed to compile this document


\usepackage{relsize,epsfig,makeidx,color,setspace,amsmath,amsfonts}
\usepackage[table]{xcolor}
\usepackage{bm,microtype}
\usepackage{fancyvrb} % packages needed for verbatim environments


%\usepackage[latin1]{inputenc}
\usepackage[utf8]{inputenc}

% Hyperlinks in PDF:
\definecolor{linkcolor}{rgb}{0,0,0.4}
\usepackage[%
    colorlinks=true,
    linkcolor=linkcolor,
    urlcolor=linkcolor,
    citecolor=black,
    filecolor=black,
    %filecolor=blue,
    pdfmenubar=true,
    pdftoolbar=true,
    bookmarksdepth=3   % Uncomment (and tweak) for PDF bookmarks with more levels than the TOC
            ]{hyperref}
%\hyperbaseurl{}   % hyperlinks are relative to this root

\setcounter{tocdepth}{2}  % number chapter, section, subsection


% prevent orhpans and widows
\clubpenalty = 10000
\widowpenalty = 10000

% http://www.ctex.org/documents/packages/layout/titlesec.pdf
\usepackage[compact]{titlesec}  % narrower section headings

% --- end of standard preamble for documents ---


% insert custom LaTeX commands...

\makeindex

%-------------------- end preamble ----------------------

\begin{document}


\renewcommand{\u}{\pmb{u}}

\newcommand{\xbm}{\bm{x}}
\newcommand{\normalvecbm}{\bm{n}}
\newcommand{\ubm}{\bm{u}}


\newcommand{\x}{\pmb{x}}
\newcommand{\normalvec}{\pmb{n}}
\newcommand{\Ddt}[1]{\frac{D#1}{dt}}
\newcommand{\halfi}{1/2}
\newcommand{\half}{\frac{1}{2}}
\newcommand{\report}{test report}

% ------------------- main content ----------------------



% ----------------- title -------------------------
\begin{center}
{\LARGE\bf
\begin{spacing}{1.25}
How various formats can deal with {\LaTeX} math
\end{spacing}
}
\end{center}

% ----------------- author(s) -------------------------

\begin{center}
{\bf HPL${}^{}$} \\ [0mm]
\end{center}

\begin{center}
% List of all institutions:
\end{center}
% ----------------- end author(s) -------------------------


\begin{center}
Aug 16, 2013
\end{center}

\vspace{1cm}



This document is translated to the format \textbf{pdflatex}. The purpose is to
test math and doconce and various output formats.

\paragraph{Test 1: Inline math.}
Here is a sentence contains the equation $u(t)=e^{-at}$.

\paragraph{Test 2: A single equation without label.}
Here it is

\[ u(t)=e^{-at} \]

\paragraph{Test 3: A single equation with label.}
Here it is as a one-line
latex code,

\begin{Verbatim}[numbers=none,fontsize=\fontsize{9pt}{9pt},baselinestretch=0.95]
!bt
\begin{equation} u(t)=e^{-at} label{eq1}\end{equation}
!et
\end{Verbatim}
looking like

\begin{equation} u(t)=e^{-at} \label{eq1}\end{equation}
and as a three-line latex code:

\begin{Verbatim}[numbers=none,fontsize=\fontsize{9pt}{9pt},baselinestretch=0.95]
!bt
\begin{equation}
u(t)=e^{-at} label{eq1b}
\end{equation}
!et
\end{Verbatim}
looking like

\begin{equation}
u(t)=e^{-at} \label{eq1b}
\end{equation}
This equation has label (\ref{eq1b}).


\paragraph{Test 4: Multiple, aligned equations without label.}
Only the align
environment is supported by other formats than {\LaTeX} for typesetting
multiple, aligned equations. The code reads

\begin{Verbatim}[numbers=none,fontsize=\fontsize{9pt}{9pt},baselinestretch=0.95]
!bt
\begin{align*}
u(t)&=e^{-at}\\ 
v(t) - 1 &= \frac{du}{dt}
\end{align*}
!et
\end{Verbatim}
and results in

\begin{align*}
u(t)&=e^{-at}\\ 
v(t) - 1 &= \frac{du}{dt}
\end{align*}

\paragraph{Test 5: Multiple, aligned equations with label.}
We use align with
labels:

\begin{Verbatim}[numbers=none,fontsize=\fontsize{9pt}{9pt},baselinestretch=0.95]
!bt
\begin{align}
u(t)&=e^{-at}
label{eq2b}\\ 
v(t) - 1 &= \frac{du}{dt}
label{eq3b}
\end{align}
!et
\end{Verbatim}
and results in

\begin{align}
u(t)&=e^{-at} \label{eq2b}\\ 
v(t) - 1 &= \frac{du}{dt} \label{eq3b}
\end{align}
We can refer to the last equations as the system (\ref{eq2b})-(\ref{eq3b}).




\paragraph{Test 6: Multiple, aligned eqnarray equations without label.}
Let us
try the old eqnarray environment.

\begin{Verbatim}[numbers=none,fontsize=\fontsize{9pt}{9pt},baselinestretch=0.95]
!bt
\begin{eqnarray*}
u(t)&=& e^{-at}\\ 
v(t) - 1 &=& \frac{du}{dt}
\end{eqnarray*}
!et
\end{Verbatim}
and results in

\begin{eqnarray*}
u(t)&=& e^{-at}\\ 
v(t) - 1 &=& \frac{du}{dt}
\end{eqnarray*}

\paragraph{Test 7: Multiple, eqnarrayed equations with label.}
We use eqnarray with
labels:

\begin{Verbatim}[numbers=none,fontsize=\fontsize{9pt}{9pt},baselinestretch=0.95]
!bt
\begin{eqnarray}
u(t)&=& e^{-at}
label{eq2c}\\ 
v(t) - 1 &=& \frac{du}{dt}
label{eq3c}
\end{eqnarray}
!et
\end{Verbatim}
and results in

\begin{eqnarray}
u(t)&=& e^{-at} \label{eq2c}\\ 
v(t) - 1 &=& \frac{du}{dt} \label{eq3c}
\end{eqnarray}
Can we refer to the last equations as the system (\ref{eq2c})-(\ref{eq3c})?

\paragraph{Test 8: newcommands and boldface bm vs pmb.}
We have

\[ \color{blue}{\frac{\partial\u}{\partial t}} +
\nabla\cdot\nabla\u = \nu\nabla^2\u -
\frac{1}{\varrho}\nabla p,\]
and $\nabla\u (\x)\cdot\normalvec$
with plain old pmb. Here are the same formulas using \Verb!\bm!:

\[ \color{blue}{\frac{\partial\ubm}{\partial t}} +
\nabla\cdot\nabla\ubm = \nu\nabla^2\ubm -
\frac{1}{\varrho}\nabla p,\]
and $\nabla\ubm (\xbm)\cdot\normalvecbm$.


% ------------------- end of main content ---------------


\printindex

\end{document}

